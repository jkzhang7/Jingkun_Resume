% !TEX TS-program = xelatex
% !TEX encoding = UTF-8 Unicode
% -*- coding: UTF-8; -*-
% vim: set fenc=utf-8

%%%%%%%%%%%%%%%%%%%%%%%%%%%%%%%%%%%%%%%%%%%%%%%%%%%%%%%%%%%%%%%%%
%% SIMPLE-RESUME-CV
%% <https://github.com/zachscrivena/simple-resume-cv>
%% This is free and unencumbered software released into the
%% public domain; see <http://unlicense.org> for details.
%%%%%%%%%%%%%%%%%%%%%%%%%%%%%%%%%%%%%%%%%%%%%%%%%%%%%%%%%%%%%%%%%

% See "README.md" for instructions on compiling this document.

\documentclass[letterpaper,MMMyyyy,nonstopmode]{simpleresumecv}
% Class options:
% a4paper, letterpaper, nonstopmode, draftmode
% MMMyyyy, ddMMMyyyy, MMMMyyyy, ddMMMMyyyy, yyyyMMdd, yyyyMM, yyyy

%%%%%%%%%%%%%%%%%%%%%%%%%%%%%%%%%%%%%%%%%%%%%%%%%%%%%%%%%%%%%%%%%
%% PREAMBLE.
%%%%%%%%%%%%%%%%%%%%%%%%%%%%%%%%%%%%%%%%%%%%%%%%%%%%%%%%%%%%%%%%%
\usepackage{titlesec}
% CV Info (to be customized).
\newcommand{\CVAuthor}{Jingkun Zhang}
\newcommand{\CVTitle}{Jingkun's CV}
\newcommand{\CVNote}{CV compiled on {\today} for Acme Corporation}
\newcommand{\CVWebpage}{http://jkzhang7.github.io}

% PDF settings and properties.
\hypersetup{
pdftitle={\CVTitle},
pdfauthor={\CVAuthor},
pdfsubject={\CVWebpage},
pdfcreator={XeLaTeX},
pdfproducer={},
pdfkeywords={},
unicode=true,
bookmarks=true,
bookmarksopen=true,
pdfstartview=FitH,
pdfpagelayout=OneColumn,
pdfpagemode=UseOutlines,
hidelinks, 
breaklinks}

% Shorthand.
\newcommand{\Code}[1]{\mbox{\textbf{#1}}}
\newcommand{\CodeCommand}[1]{\mbox{\textbf{\textbackslash{#1}}}}
 %设置页眉
%%%%%%%%%%%%%%%%%%%%%%%%%%%%%%%%%%%%%%%%%%%%%%%%%%%%%%%%%%%%%%%%%
%% ACTUAL DOCUMENT.
%%%%%%%%%%%%%%%%%%%%%%%%%%%%%%%%%%%%%%%%%%%%%%%%%%%%%%%%%%%%%%%%%

\begin{document}



%%%%%%%%%%%%%%%
% TITLE BLOCK %
%%%%%%%%%%%%%%%

\Title{\CVAuthor}

\begin{SubTitle}
\par
\href{https://www.google.com/maps/place/University+of+Washington/@47.6545874,-122.307021,15.13z/data=!4m5!3m4!1s0x549014929d8535eb:0x6b742c7901b82ba3!8m2!3d47.6553351!4d-122.3035199?shorturl=1}
{University of Washington, Seattle, WA, 98195}
\par
%\href{mailto:Jingkun_Zhang@163.com}
%{Jingkun\_Zhang@163.com}
%\,\SubBulletSymbol\,
\href{mailto:jkzhang7@uw.edu}
{jkzhang7@uw.edu}
\,\SubBulletSymbol\,
\href{mailto:jkzhang7@hotmail.com}
{jkzhang7@hotmail.com}
\,\SubBulletSymbol\,
+1\,(206)\,928-5253
\,\SubBulletSymbol\,
\href{\CVWebpage}
{\url{\CVWebpage}}
\end{SubTitle}

\begin{Body}


%%%%%%%%%%%
% Summary %%
%%%%%%%%%%%

% \Section
% {Summary}
% {Summary}
% {PDF:Summary}

% \Entryhttps
% \BulletItem
% 3 years experience of software development, robotics and machine learning.
% \BulletItem
% To apply for the 2020 summer software engineer intern position in the discipline of Autonomous Driving, Robotics and Computer Vision.
% % \BulletItem
% % To apply for the 2020 summer software engineer intern position.

%%%%%%%%%%%%%%%
%% EDUCATION %%
%%%%%%%%%%%%%%%

\Section
{Education}
{Education}
{PDF:Education}

\Entry
\href{http://www.me.washington.edu/}
{\textbf{University of Washington, Seattle (UW)}}
\hfill Seattle, WA, USA
\Gap
\BulletItem
M.Sc in
{Mechanical Engineering - Robotics}
\hfill
\DatestampYMD{2019}{09}{25} --
\DatestampYMD{2020}{12}{15} (expected)
\begin{Detail}
\end{Detail}
\BulletItem
Core Courses: Deep Learning, Computer Vision, Database Systems, Operating System
\BulletItem
Research focusing on Computer Vision and Robotics, supervised by Prof. Jenq-Neng Hwang
%\BulletItem
%Coursera: Machine Learning, Deep Learning
\Gap
\href{http://en.sjtu.edu.cn/}
{\textbf{Shanghai Jiao Tong University (SJTU)}}
\hfill Shanghai, China
\Gap
\BulletItem
B.Sc in
\href{http://202.120.53.238/English/}
{Power and Energy Engineering}
\hfill
\DatestampYMD{2014}{09}{12} --
\DatestampYMD{2018}{06}{30}
\begin{Detail}
\end{Detail}


% %%%%%%%%%%%%%%%%%%
% %% PUBLICATIONS %%
% %%%%%%%%%%%%%%%%%%

% \Section
% {Publications}
% {Publications}
% {PDF:Publications}


% \SubSection
% {Conferences}
% {Conferences}
% {PDF:Conferences}

% % Declare a new group to limit the scope of \MaxNumberedItem to this subsection.
% \begingroup
% \renewcommand{\MaxNumberedItem}{[8888]}

% \BigGap
% \NumberedItem{[1]}
% {Y.Wang, \underline{Jingkun Zhang}, X.Cheng and D.Zhang,
% ``An Assistive System for Upper Limb Motion Combining
% Functional Electrical Stimulation and Robotic Exoskeleton''
% in \textbf{The International Functional Electrical Stimulation Society},
% RehabWeek, Toronto, Canada,
% \DatestampYM{2019}{06}.}


% \endgroup

%%%%%%%%%%%%
%% SKILLS %%
%%%%%%%%%%%%

\Section
{Skills}
{Skills}
{PDF:Skills}

\Entryhttps
\BulletItem
Language: Python, C/C++, Java, SQL
\BulletItem
Software: Matlab, Simulink, Linux, LaTex, CUDA
\BulletItem
Framework: OpenCV, TensorFlow, PyTorch, ROS, Pandas, Numpy, Scikit-learn
% \BulletItem
% Adept in Mechanical Design and Embedded System Design

%%%%%%%%%%%%%%%%%%%%%%%%%
%% INDUSTRY EXPERIENCE %%
%%%%%%%%%%%%%%%%%%%%%%%%%

\Section
{Industry Experience}
{Industry Experience}
{PDF:IndustryExperience}

\Entry
\href{http://www.hyperci.com/}
{\textbf{Hyperception Technology}}
Robotics Software Engineer (C++)
\hfill
\DatestampYMD{2019}{03}{19} --
\DatestampYMD{2019}{06}{30}
\Gap

% \begin{Detail}
\BulletItem
Programmed to control the chassis system of robot vacuum cleaner based on STM32
\BulletItem
Implemented Single Camera Simultaneous localization and mapping (ORB-SLAM) algorithm for robot vacuum cleaner's autonomous driving and developed SDK for embedded LINUX system
% \end{Detail}

%%%%%%%%%%%%%%%%%%%%%%%%%
%% SELECTED PROJECTS %%
%%%%%%%%%%%%%%%%%%%%%%%%%

\Section
{Selected Projects}
{Selected Projects}
{PDF:SelectedProjects}

\Entry
{\textbf{Implement a Simple Database Management System}} (Java)
\hfill
\DatestampYMD{2020}{01}{01} --
\DatestampYMD{2020}{03}{15}
\Gap
% \begin{Detail}
\BulletItem
Built a simple database management system called SimpleDB using Java
% \end{Detail}
\Gap
{\textbf{Whale Identification – Few shot learning}} (Python)
\hfill
\DatestampYMD{2019}{11}{01} --
\DatestampYMD{2019}{12}{15}
\Gap
% \begin{Detail}
\BulletItem
Implement DenseNet and Siamese Neural Network(SNN) in Pytorch to train a deep learning model to classify the whales species by flukes image 
% \end{Detail}
\Gap
\Entry
{\textbf{Pose estimation of vehicles based on a single image }} (Python)
\hfill
\DatestampYMD{2019}{11}{01} --
\DatestampYMD{2019}{12}{15}
\Gap
% \begin{Detail}
\BulletItem
Extended Mask R-CNN neural network in Pytorch to train a deep learning model to detect the nearby cars and predict the pose in 6 degrees of freedom (Translation and Rotation) of vehicles in a single image from the Baidu's autonomous driving dataset
% \end{Detail}
\Gap


%%%%%%%%%%%%%%%%%%%%%%%%%
%% RESEARCH EXPERIENCE %%
%%%%%%%%%%%%%%%%%%%%%%%%%

\Section
{Research Experience}
{Research Experience}
{PDF:ResearchExperience}


\Entry
{\textbf{Visual Odometry and Mapping on Monocular Vision Drone}} (C++)
\hfill UW
\Gap
% \begin{Detail}
\BulletItem
Deployed semidirect visual odometry (SVO) algorithm and multi-view stereo algorithm to estimate depth map for monocular vision drone

% \end{Detail}

\Gap




\Entry
{\textbf{Control on a Novel Pick-and-place Robot}} (Matlab)
%(Research Assistant)
\hfill McGill University
% \Gap
% Advisor: Prof. Jorge Angeles, Centre for Intelligent Machines
% \hfill
% \DatestampYMD{2018}{07}{15} --
% \DatestampYMD{2018}{11}{30}

\Gap
% \begin{Detail}
\BulletItem
Literature review about the schönies motion generator (SMG) and pick-and-place robot
\BulletItem
Improved the control scheme by applying Linear–quadratic regulator controller (LQR)  and Extended Kalman Filter (EKF) to reduce trajectory-planning errors
\BulletItem
Conducted the simulation and the practical experiments to verify the control scheme
% \end{Detail}

\Gap
{\textbf{Assistive System for Upper Limb Motion, Combining Functional Electrical Stimulation and Robotic Exoskeleton}} (C)
%(Research Assistant)
\hfill SJTU
% \Gap
% Advisor: Prof. Dingguo Zhang, School of Mechanical Engineering
% \hfill
% \DatestampYMD{2017}{12}{15} --
% \DatestampYMD{2018}{05}{15}

\Gap
% \begin{Detail}
\BulletItem
Designed an embedded system for robotics glove’s control based on STM32 32-bit Arm Cortex MCU
\BulletItem
Programmed to control movement for human’s wrist joint, elbow joint and shoulder joint by Functional Electrical Stimulation and robotics exoskeleton
\BulletItem
Integrated the system for upper limb motion with 7 degrees of freedom and tested by conducting grabbing  experiments
% \end{Detail}

\Gap
% {\textbf{Energy Harvesting Device}
% %(Research Assistant)
% \hfill UW-Madison
% % \Gap
% % Advisor: Prof. Tom Krupenkin, Department of Mechanical Engineering
% % \hfill
% % \DatestampYMD{2017}{06}{15} --
% \DatestampYMD{2017}{09}{15}

% \Gap
% \begin{Detail}
% \BulletItem
% Designed and manufactured a system of energy harvesting that can transfer mechanical energy to electric energy in human locomotion, which has a much higher energy conversion rate
% \BulletItem
% Built the actual experimental setup including LabView programming and measured the conversion rate
% \BulletItem
% Improved the design by changing the structure of locomotive part
% \end{Detail}



%%%%%%%%%%%%%%%%%%
%% PUBLICATIONS %%
%%%%%%%%%%%%%%%%%%

\Section
{Publications}
{Publications}
{PDF:Publications}


\SubSection
{Conferences}
{Conferences}
{PDF:Conferences}

% Declare a new group to limit the scope of \MaxNumberedItem to this subsection.
\begingroup
\renewcommand{\MaxNumberedItem}{[8888]}

\BigGap
\NumberedItem{[1]}
{Y.Wang, \underline{Jingkun Zhang}, X.Cheng and D.Zhang,
``An Assistive System for Upper Limb Motion Combining
Functional Electrical Stimulation and Robotic Exoskeleton''
in \textbf{The International Functional Electrical Stimulation Society},
RehabWeek, Toronto, Canada,
\DatestampYM{2019}{06}.}
\endgroup

% %%%%%%%%%%%%%%%%%%%%%%%%%
% %% SELECTED PROJECTS %%
% %%%%%%%%%%%%%%%%%%%%%%%%%

% \Section
% {Selected Projects}
% {Selected Projects}
% {PDF:SelectedProjects}

% \Entry
% {\textbf{Whale species classifier by flukes image}} Kaggle Competition
% \hfill
% \DatestampYMD{2019}{11}{01} --
% \DatestampYMD{2019}{12}{15}
% \Gap
% \begin{Detail}
% \BulletItem
% Implement ResNet neural network in Pytorch to train a deep learning model to classify the whales species by flukes iamge 
% \end{Detail}
% \Gap
% \Entry
% {\textbf{Pose estimattion of vehicles based on a single image }} Kaggle Competition
% \hfill
% \DatestampYMD{2019}{11}{01} --
% \DatestampYMD{2019}{12}{15}
% \Gap
% \begin{Detail}
% \BulletItem
% Implement Mask R-CNN neural network in Pytorch to train a deep learning model to predict the pose in 6 degrees of freedom of vehicles in a single image
% \end{Detail}
% \Gap



\end{document}